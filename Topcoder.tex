\documentclass[a4j,10pt,oneside,openany]{jsbook}

%
\usepackage{amsmath,amssymb}
\usepackage{bm}
\usepackage{graphicx}
\usepackage{ascmac}
\usepackage{makeidx}
\usepackage{verbatim}
\usepackage{listings}
\usepackage{framed}
\usepackage{wrapfig}
\usepackage{fancybox}
%\usepackage{/usr/local/texlive/texmf-local/tex/latex/itembkbx/itembkbx}
\usepackage{eclbkbox}
\usepackage{itembkbx}
%
\makeindex
%
\newcommand{\diff}{\mathrm{d}}  %微分記号
\newcommand{\divergence}{\mathrm{div}\,}  %ダイバージェンス
\newcommand{\grad}{\mathrm{grad}\,}  %グラディエント
\newcommand{\rot}{\mathrm{rot}\,}  %ローテーション
%
\setlength{\textwidth}{\fullwidth}
\setlength{\textheight}{44\baselineskip}
\addtolength{\textheight}{\topskip}
\setlength{\voffset}{-0.6in}
%
\title{{\Huge \textbf{Topcoder勉強録}}\\ {\small Ver. 0.0.1}}
\author{井上裕太\\ \texttt{http://y-ino.com}}
\date{\today}
\begin{document}
\begin{abstract}
~~~~~~~Topcoder楽しいヨ
\end{abstract}
%
%
\maketitle
\frontmatter
\tableofcontents
%
%
\mainmatter
\chapter{簡単な問題}
実行時間制限が1秒未満の場合\\
$10^6$~~~~余裕をもって間に合う\\
$10^7$~~~~おそらく間に合う\\
$10^8$~~~~非常にシンプルな処理でない限り厳しい\\
\section{SRM 144}
\begin{breakitembox}[l]{Time}
{\bf Problem Statement}\\
Computers tend to store dates and times as single numbers which represent the number of seconds or milliseconds since a particular date. Your task in this problem is to write a method whatTime, which takes an int, seconds, representing the number of seconds since midnight on some day, and returns a string formatted as "H:M:S". Here, H represents the number of complete hours since midnight, M represents the number of complete minutes since the last complete hour ended, and S represents the number of seconds since the last complete minute ended. Each of H, M, and S should be an integer, with no extra leading 0's. Thus, if seconds is 0, you should return "0:0:0", while if seconds is 3661, you should return "1:1:1".\\\\
{\bf Definition}\\
Class: Time\\
Method: whatTime\\
Parameters: int\\
Returns: string\\
Method signature: string whatTime(int seconds)\\
(be sure your method is public)\\
{\bf Limits}\\
Time limit (s): 2.000\\
Memory limit (MB): 64\\
{\bf Constraints}
- seconds will be between 0 and 24*60*60 - 1 = 86399, inclusive.\\
\end{breakitembox}
\subsection*{コード}
\begin{framed}
\verbatiminput{code/Time.cpp}
\end{framed}

\subsection*{説明}
second,minute,hourの順でそれぞれの値が確定していくため、その順に計算する\\

\hrulefill

\chapter{中程度の問題}
\section{SRM 144}
\begin{breakitembox}[l]{BinaryCode}
{\bf Problem Statement}\\\\
Let's say you have a binary string such as the following:\\
011100011\\
One way to encrypt this string is to add to each digit the sum of its adjacent digits. For example, the above string would become:\\
123210122\\
In particular, if P is the original string, and Q is the encrypted string, then Q[i] = P[i-1] + P[i] + P[i+1] for all digit positions i. Characters off the left and right edges of the string are treated as zeroes.\\\\
An encrypted string given to you in this format can be decoded as follows (using 123210122 as an example):\\\\
1. Assume P[0] = 0.\\
2. Because Q[0] = P[0] + P[1] = 0 + P[1] = 1, we know that P[1] = 1.\\
3. Because Q[1] = P[0] + P[1] + P[2] = 0 + 1 + P[2] = 2, we know that P[2] = 1.\\
4. Because Q[2] = P[1] + P[2] + P[3] = 1 + 1 + P[3] = 3, we know that P[3] = 1.\\
5. Repeating these steps gives us P[4] = 0, P[5] = 0, P[6] = 0, P[7] = 1, and P[8] = 1.\\
6. We check our work by noting that Q[8] = P[7] + P[8] = 1 + 1 = 2. Since this equation works out, we are finished, and we have recovered one possible original string.\\\\
Now we repeat the process, assuming the opposite about P[0]:\\\\
1. Assume P[0] = 1.\\
2. Because Q[0] = P[0] + P[1] = 1 + P[1] = 0, we know that P[1] = 0.\\
3. Because Q[1] = P[0] + P[1] + P[2] = 1 + 0 + P[2] = 2, we know that P[2] = 1.\\
4. Now note that Q[2] = P[1] + P[2] + P[3] = 0 + 1 + P[3] = 3, which leads us to the conclusion that P[3] = 2. However, this violates the fact that each character in the original string must be '0' or '1'. Therefore, there exists no such original string P where the first digit is '1'.\\
Note that this algorithm produces at most two decodings for any given encrypted string. There can never be more than one possible way to decode a string once the first binary digit is set.\\
Given a String message, containing the encrypted string, return a String[] with exactly two elements. The first element should contain the decrypted string assuming the first character is '0'; the second element should assume the first character is '1'. If one of the tests fails, return the string "NONE" in its place. For the above example, you should return {"011100011", "NONE"}.\\

{\bf Definition}\\\\
Class:	BinaryCode\\
Method:	decode\\
Parameters:	String\\
Returns:	String[]\\
Method signature:	String[] decode(String message)\\
(be sure your method is public)\\

{\bf Constraints}\\\\
-	message will contain between 1 and 50 characters, inclusive.\\
-	Each character in message will be either '0', '1', '2', or '3'.\\

\end{breakitembox}
\subsection*{コード}
\begin{framed}
\verbatiminput{code/BinaryCode.cpp}
\end{framed}

\subsection*{説明}
$a_1$から順に加えるかどうかを決めていき、n個全てについて決め終わったらその和がkに等しいかを判定する

\hrulefill


\chapter{難しい問題}
\section{SRM 144}
\begin{breakitembox}[l]{BinaryCode}
{\bf Problem Statement}\\\\
You work for an electric company, and the power goes out in a rather large apartment complex with a lot of irate tenants. You isolate the problem to a network of sewers underneath the complex with a step-up transformer at every junction in the maze of ducts. Before the power can be restored, every transformer must be checked for proper operation and fixed if necessary. To make things worse, the sewer ducts are arranged as a tree with the root of the tree at the entrance to the network of sewers. This means that in order to get from one transformer to the next, there will be a lot of backtracking through the long and claustrophobic ducts because there are no shortcuts between junctions. Furthermore, it's a Sunday; you only have one available technician on duty to search the sewer network for the bad transformers. Your supervisor wants to know how quickly you can get the power back on; he's so impatient that he wants the power back on the moment the technician okays the last transformer, without even waiting for the technician to exit the sewers first.\\
You will be given three vector <int>'s: fromJunction, toJunction, and ductLength that represents each sewer duct. Duct i starts at junction (fromJunction[i]) and leads to junction (toJunction[i]). ductlength[i] represents the amount of minutes it takes for the technician to traverse the duct connecting fromJunction[i] and toJunction[i]. Consider the amount of time it takes for your technician to check/repair the transformer to be instantaneous. Your technician will start at junction 0 which is the root of the sewer system. Your goal is to calculate the minimum number of minutes it will take for your technician to check all of the transformers. You will return an int that represents this minimum number of minutes.\\
{\bf Definition}\\\\
Class: PowerOutage\\
Method: estimateTimeOut\\
Parameters: vector <int>, vector <int>, vector <int>\\
Returns: int\\
Method signature: int estimateTimeOut(vector <int> fromJunction, vector <int> toJunction, vector <int> ductLength)\\
(be sure your method is public)\\

{\bf Constraints}\\\\
- fromJunction will contain between 1 and 50 elements, inclusive.\\
- toJunction will contain between 1 and 50 elements, inclusive.\\
- ductLength will contain between 1 and 50 elements, inclusive.\\
- toJunction, fromJunction, and ductLength must all contain the same number of elements.\\
- Every element of fromJunction will be between 0 and 49 inclusive.\\
- Every element of toJunction will be between 1 and 49 inclusive.\\
- fromJunction[i] will be less than toJunction[i] for all valid values of i.\\
- Every (fromJunction[i],toJunction[i]) pair will be unique for all valid values of i.\\
- Every element of ductlength will be between 1 and 2000000 inclusive.\\
- The graph represented by the set of edges (fromJunction[i],toJunction[i]) will never contain a loop, and all junctions can be reached from junction 0.\\

\end{breakitembox}
\subsection*{コード}
\begin{framed}
\verbatiminput{code/PowerOutage.cpp}
\end{framed}

\subsection*{説明}

\hrulefill

\newpage
\printindex
%
%
\end{document}